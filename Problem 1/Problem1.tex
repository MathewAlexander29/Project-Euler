% --------------------------------------------------------------
% This is all preamble stuff that you don't have to worry about.
% Head down to where it says "Start here"
% --------------------------------------------------------------
 
\documentclass[12pt]{article}
 
\usepackage[margin=1in]{geometry}
\usepackage{amsmath,amsthm,amssymb}
\usepackage{mathtools}
 
\newcommand{\N}{\mathbb{N}}
\newcommand{\Z}{\mathbb{Z}}

\DeclarePairedDelimiter\ceil{\lceil}{\rceil}
\DeclarePairedDelimiter\floor{\lfloor}{\rfloor}
 
\newenvironment{theorem}[2][Theorem]{\begin{trivlist}
\item[\hskip \labelsep {\bfseries #1}\hskip \labelsep {\bfseries #2.}]}{\end{trivlist}}
\newenvironment{lemma}[2][Lemma]{\begin{trivlist}
\item[\hskip \labelsep {\bfseries #1}\hskip \labelsep {\bfseries #2.}]}{\end{trivlist}}
\newenvironment{exercise}[2][Exercise]{\begin{trivlist}
\item[\hskip \labelsep {\bfseries #1}\hskip \labelsep {\bfseries #2.}]}{\end{trivlist}}
\newenvironment{problem}[2][Problem]{\begin{trivlist}
\item[\hskip \labelsep {\bfseries #1}\hskip \labelsep {\bfseries #2.}]}{\end{trivlist}}
\newenvironment{question}[2][Question]{\begin{trivlist}
\item[\hskip \labelsep {\bfseries #1}\hskip \labelsep {\bfseries #2.}]}{\end{trivlist}}
\newenvironment{corollary}[2][Corollary]{\begin{trivlist}
\item[\hskip \labelsep {\bfseries #1}\hskip \labelsep {\bfseries #2.}]}{\end{trivlist}}

\newenvironment{solution}{\begin{proof}[Solution]}{\end{proof}}
 
\begin{document}
 
% --------------------------------------------------------------
%                         Start here
% --------------------------------------------------------------
 
\title{Project Euler}%replace X with the appropriate number
\author{Mathew Alexander\\ %replace with your name
Problem 1} %if necessary, replace with your course title
 
\maketitle
 
\begin{problem}{1}
If we list all natural numbers below 10 that are multiples of 3 or 5, we get 3,5,6 and 9. The sum of these multiples is 23.
Find the sum of all multiples of 3 or 5 \textbf{below} 1000.
\end{problem}

\begin{solution}
This is a simple summation of arithmetic progression problem \dots
\\\\Consider the series
\\\\3, 6, 9, 12, 15, 18, 21, 24, 27, 30, 33 \dots last number, where the last number is unknown. But we know that it is a multiple of 3 and is below 1000.
\\\\Sum of n numbers in an arithmetic progression(AP) is given by the equation
\\ \[S_n = \frac{n}{2}(2a+(n-1)d)\]
\\where 
\\ \textit{ a = first term in the AP
\\n = number of terms in AP
\\d = difference between terms in an AP
}
\\
\\If the last term is known then the equation becomes
\\\[S_n = \frac{n}{2}(a + l)\]
\\where 
\\ \textit{ l = last term in the AP}
\\\\Hence \dots 
\\ \[ \floor{\frac{1000}{3}} = 999\] with 1 as remainder
\\Also \[\frac{999}{3} = 333\]
\\Here for the series of multiples of 3,
\textbf{
\\ a =3 ,
\\l = 999,
\\n = 333
}
\\
\\
\textit{llly...}
\\For the series of multiples of 5
\textbf{
\\ a =5 ,
\\l = 995,
\\n = 199
}
\\
\\Here there is another thing we have to be careful about. As we are summing multiples of 3 or 5, the multiple of 3 and 5(i.e multiples of 15) will be added twice. Hence the same has to be subtracted from total to obtain the correct sum.
\\For the series of multiples of 15
\textbf{
\\ a =15 ,
\\l = 990,
\\n = 66
}
\\
\\So the final solution is
\\\[Sum = S_3 + S_5 - S_{15}\]

\end{solution} 


\end{document}
